%\usepackage[draft]{hyperref}
%\newcommand{\PE}{{\mathcal P}(x)}
%\newcommand{\PEP}{{\mathcal P}[x]}
%\usepackage{epsfig}
%\usepackage{pstricks}
%\usepackage{pst-node}
%\usepackage{pst-tree}
%\usepackage{pst-grad}
%\usepackage{pst-plot}
%\DeclareMathOperator{\IntegralElementSimple}{IntegralElementSimple}
%\DeclareMathOperator{\LocalIntegralBasisElement}{LocalIntegralBasisElement}

\documentclass[a4paper,11pt]{amsart}%
\usepackage{multicol}
\usepackage{bm}
\usepackage[utf8x]{inputenc}
\usepackage{amssymb}
\usepackage{amsmath}
\usepackage{synttree}
\usepackage{amsthm}
\usepackage{algorithm}
%\usepackage{algpseudocode}
\usepackage[noend]{algorithmic}
\usepackage{todonotes}
\usepackage{amsfonts}
\usepackage{graphicx}
\usepackage{comment}
\usepackage{hyperref}%
\usepackage{enumitem}
\setcounter{MaxMatrixCols}{30}
%TCIDATA{OutputFilter=latex2.dll}
%TCIDATA{Version=5.00.0.2570}
%TCIDATA{LastRevised=Wednesday, February 26, 2014 20:49:07}
%TCIDATA{<META NAME="GraphicsSave" CONTENT="32">}
%TCIDATA{<META NAME="SaveForMode" CONTENT="1">}
\allowdisplaybreaks
\renewcommand{\algorithmicrequire}{\textbf{Input:}}
\renewcommand{\algorithmicensure}{\textbf{Output:}}
\theoremstyle{definition}
\theoremstyle{plain}
\newtheorem{defn}{Definition}[section]
\newtheorem{theorem}[defn]{Theorem}
\newtheorem{proposition}[defn]{Proposition}
\newtheorem{lemma}[defn]{Lemma}
\theoremstyle{remark}
\newtheorem{remark}[defn]{Remark}
\newtheorem{example}[defn]{Example}
\newtheorem{notation}[defn]{Notation}
\DeclareMathOperator{\LocalIntegralBasis}{LocalIntegralBasis}
\DeclareMathOperator{\IntegralElement}{IntegralElement}
\DeclareMathOperator{\IntegralBasisIterative}{\mathtt{IntegralBasisIterative}}
\DeclareMathOperator{\TruncatedFactor}{\mathtt{IntegralBasisElement}}
\DeclareMathOperator{\Splitting}{Splitting}
\DeclareMathOperator{\BlockSplitting}{BlockSplitting}
\DeclareMathOperator{\SegmentSplitting}{SegmentSplitting}
\DeclareMathOperator{\Spec}{Spec}
\DeclareMathOperator{\Sing}{Sing}
\DeclareMathOperator{\Ann}{Ann}
\DeclareMathOperator{\Int}{Int}
\DeclareMathOperator{\Hom}{Hom}
\DeclareMathOperator{\Id}{Id}
\DeclareMathOperator{\degree}{degree}
\DeclareMathOperator{\rad}{rad}
\DeclareMathOperator{\Ker}{ker}
\DeclareMathOperator{\TQR}{Q}
\newcommand{\singular}{{\sc Singular}}
\newcommand{\maple}{{\sc Maple}}
\newcommand{\cc}{{\mathbf c}}
\newcommand{\Q}{{\mathbb Q}}
\newcommand{\N}{{\mathbb N}}
\newcommand{\R}{{\mathbb R}}
\newcommand{\Z}{{\mathbb Z}}
\newcommand{\C}{{\mathbb C}}
\newcommand{\Px}{{\mathcal{P}_X}}

\usepackage{accents}
\newcommand{\dbtilde}[1]{\accentset{\approx}{#1}}
\newcommand{\vardbtilde}[1]{\tilde{\raisebox{0pt}[0.85\height]{$\tilde{#1}$}}}


\begin{document}
\title[Good truncations for computing integral bases]{Good truncations for computing integral bases}
\author{D. Basson, J. Boehm, S. Laplagne, M. Marais}

\begin{abstract}
We provide good truncation bound that speed up the computation of integral basis.
\end{abstract}
\maketitle

\section{Introduction}
\label{section:introduction}

Integral basis are very useful in real life.

Given a polynomial $f \in K[x,y]$ monic in $Y$, we wish to compute the local contribution to the integral basis at the origin or the local integral basis at the origin.


\section{The determinacy}
\label{section:directApproach}

In this section we show:
\begin{enumerate}
\item An upper bound for the determinacy can be found fast reducing the polynomial modulo any prime number $p$.
\item If we truncate the polynoial $f$ by standard degree at the determinacy, the characteristic exponents of $f$ will not change.
\item The integrality exponent of $f$ can be computed from the characteristic exponents, so we can compute very fast the integrality exponent.
\end{enumerate}

Remark: the coefficients of the Puiseux expansions might change, but in all examples we tried, the coefficients don't change after truncation.

\section{Local contribution to the integral basis at the origin}

In this section we focus on the computation of the local contribution at the origin.

In this case, by \cite[Proposition 23]{poteaux2021} we can truncate the powers of $X$ at order two times the integrality exponent and we will get the correct integral basis.

\section{Local integral basis at the origin}

When we are interested in the local integral basis at the origin we can also truncate powers of $Y$.

Applying \cite[Proposition 23]{poteaux2021}  in terms of $Y$ we obtain that the factors at the origin are uniquely determined and they are the same factors as the factors obtained when we develope the factorization in terms of $X$, so we are able to recover the correct information.

The factor $g_0$ however corresponds to the branches of $f$ at $Y = 0$ and the factor $f_0$ corresponds to the branches of $f$ at $X = 0$, so we cannot recover the Puiseux expansions of $f_0$ from the Puiseux expansions of $g_0$.

Hence when we apply a truncation in $Y$ we can expect to compute correctly the local integral basis but not the local contribution.



\section{Timings}

We get very good timings.

\bibliographystyle{plain}
\bibliography{mybib}



\end{document}


%\begin{remark}
%We always can find an integral basis of type
%\[
%1,\frac{p_{1}}{x^{e_{1}}},\dots,\frac{p_{n-1}}{x^{e_{n-1}}}.
%\]
%\todo{fix this: if only singularities with $x = 0$ occur. Move to a later section.}
%\end{remark}

\begin{remark}
Let $L=\overline{K}$ and let $A=L[x,y]=L[X,Y]/\left\langle f\right\rangle $ be
denote the coordinate ring of $C$ with $f\in K[X][Y]$ monic in $Y$ of degree
$n$. Let $a_{1},...,a_{s}\in L$ denote the $x$-coordinates of the
singularities of $C$. Then there are $p_{i}\in K[X][Y]$ with $\deg_{Y}%
(p_{i})=i$ for all $i$ and $e_{i,j}\in\mathbb{N}_{0}$ with $e_{i,j}\geq
e_{i-1,j}$, such that, as an $L[X]$-module%
\[
\bar{A}=_{L[X]}\left\langle \frac{\bar{p}_{0}}{\bar{q}_{0}},\ldots,\frac
{\bar{p}_{n-1}}{\bar{q}_{n-1}}\right\rangle \subseteq
Q(A)=L(X)[Y]/\left\langle f\right\rangle
\]
with%
\[
q_{i}=%
%TCIMACRO{\dprod \limits_{j=1}^{s}}%
%BeginExpansion
{\displaystyle\prod\limits_{j=1}^{s}}
%EndExpansion
(X-a_{j})^{e_{i,j}}\in K[X]\text{.}%
\]


If $a_{1}=0$ and $P=\left\langle x,y\right\rangle $ then, as an
$L[X]_{\left\langle X\right\rangle }$-module,%
\[
\overline{A_{P}}=\left\langle \frac{\bar{p}_{0}}{x^{e_{0,1}}},\ldots
,\frac{\bar{p}_{n-1}}{x^{e_{n-1,1}}}\right\rangle
\]
and%
\[
\overline{\widehat{A_{P}}}=\left\langle \frac{\bar{p}_{0}}{x^{e_{0,1}}}%
,\ldots,\frac{\bar{p}_{n-1}}{x^{e_{n-1,1}}}\right\rangle \subseteq
Q(A_{P})\otimes_{A_{P}}\widehat{A_{P}}%
\]
as an $L[[X]]$-module.
\end{remark}

\begin{proof}
By van Hoeij's algorithm the claim for $\bar{A}$ is clear. The second claim
follows since $\_\otimes_{A}A_{P}$ is right exact, $\overline{A_{P}}%
=\overline{A}_{P}$, and $x-a$ is a unit in $\overline{A_{P}}$ if $a\neq0$. The
third claim follows since $\_\otimes_{A_{P}}\widehat{A_{P}}$ is right exact
and $\widehat{\overline{A_{P}}}=\overline{\widehat{A_{P}}}$, since the
semilocal ring $\overline{A_{P}}$ is excellent.
\end{proof}




%% Puiseux blocks with more than one class

\begin{remark}
As we have already mentioned, when $f$ has more than one conjugacy class of expansions in a Puiseux block, Algorithm \ref{algo:iterative} cannot be applied to determine which expansions to choose in each class.
The reason why this algorithm does not work is that if $\Gamma$ and $H$ are two conjugacy classes in the same Puiseux block, the order of $\gamma - \eta$, $\gamma \in \Gamma$ and $\eta \in H$, depends on which expansion we have chosen in each class.
\end{remark}

\begin{example}
Consider a polynomial $f$ with Puiseux expansions
\begin{align*}
\gamma_1 &= x^{3/2} + x^2 + \dots \\
\gamma_2 &= - x^{3/2} + x^2 + \dots \\
\gamma_3 &= x^{3/2} + x^3 + \dots \\
\gamma_4 &= -x^{3/2} + x^3 + \dots
\end{align*}
where $\{\gamma_1, \gamma_2\}$ is a conjugacy class and $\{\gamma_3, \gamma_4\}$ is another conjugacy class, and both classes are in the same block.

Then
$$\gamma_1 - \gamma_3 = x^2 - x^3 + \dots$$
has order 2 but
$$\gamma_1 - \gamma_4 = 2 x^{3/2} + x^2 - x^3 + \dots$$
has order 3/2.
\end{example}

\begin{remark}
For this case we propose a different algorithm using polynomials over the ground field as building blocks.
%, instead of considering the number of Puiseux expansions as before.
This algorithm can be applied in all cases, and in practice it is very similar to Algorithm \ref{algo:exhaustive}. However, with this approach we cannot apply an optimized strategy as in Algorithm \ref{algo:iterative} and hence it can be slow when there are many conjugacy classes.
\end{remark}

